\documentclass[11pt]{article}
\usepackage[utf8]{inputenc}

% --- General Setup --- %
\usepackage[margin=1in]{geometry}
\usepackage{float}
\usepackage{gensymb}

% --- Works Cited --- %
\usepackage[backend=biber,style=ieee,sorting=none]{biblatex}
\addbibresource{./sources.bib}

% --- Dummy Text --- %
\usepackage{lipsum}

% --- Introduction Formatting --- %
\renewcommand{\abstractname}{\Large{Introduction}}

% --- Tables --- %
\usepackage{multirow}
\usepackage{array}

\begin{document}
% ---------------- %
\begin{center}
    \large
    \textbf{GPGN 470A Term Project - Part II}
    
    \vspace{0.4cm}
    \large
    Engineering Literature Review of Delayed Doppler Map Instrumentation on board CYGNSS
    
    \vspace{0.4cm}
    Tyler Singleton
    
    \vspace{0.4cm}
    \today
    
    \vspace{0.9cm}
\end{center}
\begin{abstract}
    \normalsize
    \vspace{1em}
    
    The Cyclone Global Navigation Satellite Systems (CYGNSS) is a constellation consisting of eight micro-satellites originally intended to measure ocean wind speeds utilizing reflected GNSS signals from Global Positioning Satellites (GPS) \cite{NASA}. Recently, these reflected GNSS signals have shown potential to monitor global soil moisture (SM) levels as the signals are more sensitive to changes in SM than surface roughness and vegetation \cite{Global_SM}. The Delayed Doppler Mapping Instrument (DDMI) or SGR-ReSI is the primary instrumentation utilized to map the forward scattered GNSS signals \cite{DDMI_Overview}; it is capable of processing up to four reflections simultaneously and in real-time \cite{DDMI_Overview, DDMI_Summary}. In this paper, I will discuss an overview of the DDMI -- its history, capabilities, how it works, and display a delayed doppler map pulled from CYNGSS using level 1 data. 

\end{abstract}

% ---------------- %
\section{An Overview of Delayed Doppler Mapped Instrumentation}


% ---------------- %
\section{Data Processing}

% ---------------- %

\clearpage
\appendix
\renewcommand{\theequation}{\thesection.\arabic{equation}}
\section{Instrumentation}

\setlength{\tabcolsep}{20pt}
\renewcommand{\arraystretch}{1.25}

\begin{table}[H]
    \centering
    \begin{tabular}{>{\bfseries}p{6cm} p{6cm}}
        \multicolumn{2}{c}{Delay Doppler Mapping Instrument (DDMI)} \\
        \hline
        Instrument Technology & GNSS receiver \\ 
        Resolution & 20-50 km \\
        Geometry & Push-broom scanning \\
        Waveband & 1.575 \\
        Data Format & NetCDF \\
        Data Access & Open \\
        \hline
    \end{tabular}
    \caption{Caption \cite{DDMI_Summary}}
    \label{tab:DDMI_Summary}
\end{table}

% Sources
\clearpage
\printbibliography

\end{document}
